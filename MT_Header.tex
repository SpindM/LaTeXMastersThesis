%%%%%%%%%%%%%%%%%%%%%%%%%%%%%%%%%%%%%%%%%%%%%%%%%%
%%%%% HEADER: Specify package options %%%%%%%%%%%%
%%%%%%%%%%%%%%%%%%%%%%%%%%%%%%%%%%%%%%%%%%%%%%%%%%
%%% google packages you might not know or want to remove. 

\documentclass[12pt,a4paper]{article}
\setcounter{tocdepth}{4}

\usepackage[utf8]{inputenc}
\usepackage{tocbibind}  % to bind sections to toc (also acknowledgements etc) [nottoc] for not putting TOC in TOC
\usepackage{graphicx}
\usepackage[inline]{enumitem} % for in-line hypotheses (so you can enumerate within text).
\usepackage[acronym,nonumberlist,toc,nopostdot,nogroupskip]{glossaries} %for acronyms
\usepackage{textcomp} % for using symbols.
\usepackage[left=3.5cm,right=3cm,top=2.5cm,bottom=3cm]{geometry} %sets your page margins 
\usepackage[natbibapa]{apacite} % citation style
\usepackage{fancyhdr}       % header info
\pagestyle{fancy}
\fancyhead[R]{\normalfont\thepage}
\fancyhead[L]{}
%\renewcommand{\headrulewidth}{0pt}
\fancyfoot{}
\setlength{\headheight}{15pt}%
\setlength\parindent{0pt}
\graphicspath{ {figures/} }     % needed for list of tables/ figures
\usepackage{array}              % needed for list of tables/ figures
\usepackage[onehalfspacing]{setspace} % spacing between lines
\usepackage[hidelinks]{hyperref} % for links. if you remove "hidelinks", you get a colored border around links and references
\usepackage{xcolor} % to mark text that needs to be edited. not needed in final version
\usepackage{makecell} % for multiple line cells in tables
\usepackage{wasysym} % for male/female symbols
\usepackage{gensymb} % for commands like \degree (for MRI description)
\usepackage[german,english]{babel} % your supported languages. 
\usepackage{appendix}
\usepackage[tableposition=top,labelfont=bf,font=small]{caption} %captions are used for figures and tables.
\captionsetup{justification=raggedright,singlelinecheck=false}
\usepackage[labelformat=simple,labelfont=bf]{subcaption}
\renewcommand{\thesubtable}{\Alph{subtable}} %used for subtables (if you want to have two tables 1A and 1B for one overall table 1)
%\usepackage{multibib} % if you want to have multiple bibliographies
%\newcites{refs}{R Packages}
\usepackage{siunitx} %for placing numbers with units in text. (nicer spacing)
\usepackage{pdfpages} %for appendix with pdfs
\usepackage{floatrow}
\floatsetup[table]{capposition=top}

%%% for pdfs in appendix:
\makeatletter
\NewDocumentCommand\headerspdf{ O {pages=-} m }{% [options for include pdf]{filename.pdf}
  \includepdf[%
    #1,
    pagecommand={\thispagestyle{fancy}},
    scale=.7,
    ]{#2}}
\NewDocumentCommand\secpdf{somO{1}m}{% [short title]{section title}[page specification]{filename.pdf} --- possibly starred
  \clearpage
  \thispagestyle{fancy}%
  \includepdf[%
    pages=#4,
    pagecommand={%
     \IfBooleanTF{#1}{%
       \section{#3}}{%
       \IfNoValueTF{#2}{%
         \section{#3}}{%
         \section[#2]{#3}}}%
   },
    scale=.65,
    ]%
    {#5}
    \addtocounter{section}{-1}%
    \refstepcounter{section}%
  }%
\makeatother
%---------------------------------------------------------------

%\usepackage{fontspec}
%%% using times new roman: 
% is a non-free font, therefore you need to include the .ttf files for that (look for that on your machine, if you e.g. have photoshop, there are included in there, but i am not allowed to distribute them). if you dont want to use times, delete this part here.
%\setmainfont[Ligatures=TeX,BoldFont=TIMESBD.TTF,
%ItalicFont=TIMESI.TTF,
%BoldItalicFont=TIMESBI.TTF,]{TIMES.TTF}


\title{TITLE TITLE TITLE}								% Title
\author{Your Name}								% Author
\date{\today}	
\makeatletter
\let\thetitle\@title
\let\theauthor\@author
\let\thedate\@date
\makeatother

